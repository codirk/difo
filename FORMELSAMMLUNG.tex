% https://www.grund-wissen.de/informatik/latex/mathematischer-formelsatz.html
% https://docplayer.org/28012320-Mathematische-formelsammlung.html
% add CC license
% 

% Autor:
\author{Dirk~Messetat}
% Datum: heute
\date{\today}
% Versionsnummer:
\newcommand{\versionsnummer}{1.0.0-SNAPSHOT}

% URL:
\newcommand{\url}{}

% Titel:
\title{Mathematische \\
FORMELSAMMLUNG \\[2ex] \small Version: \versionsnummer \\[2ex] \texttt{ \url{http://github.com/codirk/formelsamlung} }}
%
% DIN A5 Seite:
\documentclass[8pt,a4paper,fleqn]{article}
% Seitengeometrie festlegen:

\setlength{\parindent}{0in}
\setlength{\mathindent}{0pt}

\usepackage[utf8]{inputenc}

\usepackage{helvet}
\usepackage{ngerman}
\usepackage{curves}
\usepackage{latexsym} % ein paar Symbole
\usepackage{textcomp} % ein paar Symbole
\usepackage[dvips]{rotating} % für rotate-Befehl
\usepackage{geometry}
\geometry{left=1.5cm,textwidth=19cm,top=0.5cm,textheight=26cm}


%\usepackage{creativecommons}

\usepackage{amsmath}
%\usepackage[draft]{graphics} % ohne Bilder (Entwurf)
\usepackage{graphics} % Bilder einbinden

\nonfrenchspacing
\renewcommand{\familydefault}{\sfdefault}


\begin{document}
\maketitle
\thispagestyle{empty}


% Fill with blanks to bottom of first page
\vfill
 

\begin{flushright}
    \copyright  2021 Dirk Messetat git [at] messetat [dot] com. \\
    This work is licensed under a Creative Commons Attribution- ShareAlike 3.0 License.
    To view a copy of this license visit:
     \url{http://creativecommons.org/licenses/by-sa/3.0/legalcode}.
\end{flushright}

\pagebreak



% Inhaltsverzeichnis
\tableofcontents
\thispagestyle{empty}


\vfill

\begin{center}
\small{Dieses Dokument wurde mit \LaTeX{} gesetzt.}
\end{center}

%
\newpage
%

\clearpage
\pagenumbering{arabic} 

\section{Mengenlehre}
\section{Arithmetik}
\subsection{Brüche}

Addition und Subtraction gleichnamiger Brüche:

\begin{align*}
  \frac{a}{c} \pm \frac{b}{c} =  \frac{a \pm b}{c} 
\end{align*}

Addition und Subtraction nicht gleichnamiger Brüche:

\begin{align*}
\frac{a}{b} \pm \frac{c}{d} = \frac{ad \pm bc}{db} 
\end{align*}

\begin{align*}
\frac{a}{b} \pm c = \frac{a}{b} \pm \frac{c}{1} = \frac{a \pm bc}{b} 
\end{align*}

Multiplication Bruch mit ganzer Zahl:
\begin{align*}
\frac{a}{b} \cdot c = \frac{a \cdot c}{b} 
\end{align*}

Division Bruch mit ganzer Zahl:
\begin{align*}
\frac{a}{b} : c = \frac{a}{b \cdot c} 
\end{align*}

Division Bruch mit Bruch:
\begin{align*}
\frac{a}{b} : \frac{c}{d} = \frac{a}{b}  \cdot \frac{d}{c} = \frac{a \cdot d }{b \cdot c} 
\end{align*}

Kürzen:
\begin{align*}
\frac{ac}{bc} = \frac{a \cdot c}{b \cdot c} =  \frac{a \cdot c : c }{b \cdot c : c} =\frac{a}{b} 
\end{align*}

Erweitern:
\begin{align*}
\frac{a}{b} =  \frac{ac}{bc} =   \frac{a \cdot c}{b \cdot c} 
\end{align*}

\subsection{Klammerrechnung}
\subsection{Binomische Formeln}
\section{Algebra}
\section{Relationen und Funktionen}
\section{Folgen und Reihen}
\section{Differenzialrechnung}
\section{Integralrechnung}
\section{Geometrie}
\subsection{Planimetrie}
\subsection{Stereometrie}
\subsection{Trigonometrie}
\subsection{Geniometrie}
\section{Vectorrechung}
$\vec{B}$
\section{Matrixrechnung}
\section{Statistik}

\end{document}
% ======================= END

$ \to $ % daraus folgt

$$\frac{a \pm b}{c} =  \frac{a}{c} \pm \frac{b}{c} $$

$\frac{a \pm b}{c} =  \frac{a}{c} \pm \frac{b}{c} $



