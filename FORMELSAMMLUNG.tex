% https://www.grund-wissen.de/informatik/latex/mathematischer-formelsatz.html
% https://docplayer.org/28012320-Mathematische-formelsammlung.html
% add CC license
% use Command+Shift+Click in the PDF to jump to the source code line when using TeXShop

% Autor:
\author{Dirk~Messetat}
% Datum: heute
\date{\today}
% Versionsnummer:
\newcommand{\versionsnummer}{1.0.0-SNAPSHOT}

% URL:
\newcommand{\url}{}

% Titel:
\title{Mathematische \\
FORMELSAMMLUNG \\[2ex] \small Version: \versionsnummer \\[2ex] \texttt{ \url{http://github.com/codirk/formelsamlung} }\\
 \hfill\break
Klasse 5-13 (auch für Bayern ;-)
} 
%
% DIN A5 Seite:
\documentclass[12pt,a4paper,fleqn,twoside,pdf,final]{article}
% Seitengeometrie festlegen:

\setlength{\parindent}{0in}
\setlength{\mathindent}{0pt}


\usepackage{amsfonts} 


\usepackage[utf8]{inputenc}

\usepackage{helvet}
\usepackage{ngerman}
\usepackage{curves}
\usepackage{latexsym} % ein paar Symbole
\usepackage{textcomp} % ein paar Symbole
\usepackage[dvips]{rotating} % für rotate-Befehl
\usepackage{geometry}
\geometry{left=0.5cm,textwidth=19cm,top=1.0cm,textheight=26cm}

\usepackage{array}


%\usepackage{creativecommons}

% striked out text
\usepackage[normalem]{ulem}
\usepackage{cancel}



\usepackage{amsmath}
%\usepackage[draft]{graphics} % ohne Bilder (Entwurf)
\usepackage{graphics} % Bilder einbinden

\nonfrenchspacing
\renewcommand{\familydefault}{\sfdefault}


% newcommand
\newcommand{\stkout}[1]{\ifmmode\text{\cancel{\ensuremath{#1}}}\else\sout{#1}\fi}



\begin{document}
\maketitle
\thispagestyle{empty}


% Fill with blanks to bottom of first page
\vfill
 

\begin{flushright}
    \copyright  2021 Dirk Messetat git [at] messetat [dot] com. \\
    This work is licensed under a Creative Commons Attribution- ShareAlike 3.0 License.
    To view a copy of this license visit:
     \url{http://creativecommons.org/licenses/by-sa/3.0/legalcode}.
\end{flushright}

\pagebreak



% Inhaltsverzeichnis
\setcounter{tocdepth}{1}
\tableofcontents
\thispagestyle{empty}


\vfill

\begin{center}
\small{Dieses Dokument wurde mit \LaTeX{} gesetzt.}
\end{center}

%
\newpage
%

\clearpage
\pagenumbering{arabic} 



\section{Mengenlehre}

% \begin{align*}
% [a;b] = TODO \{x| a <  x \leq b, x \in \mathbb{R}\} \\
% ]a;b[ = TODO \{x| a <  x \leq b, x \in \mathbb{R}\} \\
% [a;b[ = TODO \{x| a <  x \leq b, x \in \mathbb{R}\} \\
% ]a;b] = TODO \{x| a <  x \leq b, x \in \mathbb{R}\}
% \end{align*}

\section{Arithmetik}
\subsection{Brüche}

Kürzen:
\begin{align*}
\frac{ac}{bc} = \frac{a \cdot c}{b \cdot c} =  \frac{a \cdot c : c }{b \cdot c : c} =  \frac{a \cdot 1 }{b \cdot 1 } =  \frac{a \cdot \stkout{c} }{b \cdot \stkout{c}}  =  \frac{a}{b}  
\end{align*}

Erweitern:
\begin{align*}
\frac{a}{b} =  \frac{ac}{bc} =   \frac{a \cdot c}{b \cdot c} 
\end{align*}

Addition und Subtraktion gleichnamiger Brüche:

\begin{align*}
  \frac{a}{c} \pm \frac{b}{c} =  \frac{a \pm b}{c} 
\end{align*}

Addition und Subtraktion nicht gleichnamiger Brüche:

\begin{align*}
\frac{a}{b} \pm \frac{c}{d} = \frac{ad}{bd} \pm \frac{cb}{db} = \frac{ad \pm bc}{db} \longrightarrow \frac{a}{b} \pm c = \frac{a}{b} \pm \frac{c}{1} = \frac{a \pm bc}{b} 
\end{align*}

Gemischte Brüche
\begin{align*}
  c\frac{a}{b} = c \cdot \frac{a}{b}= \frac{a}{b} \cdot c= \frac{a}{b}c
\end{align*}
Achtung! : Addition bei Zahlen
\begin{align*}
2\frac{1}{3} = 2+\frac{1}{3} =  \frac{6}{3} + \frac{1}{3} = \frac{7}{3} 
\end{align*}


Multiplikation Bruch mit ganzer Zahl:

\begin{align*}
\frac{a}{b} \cdot c = \frac{a}{b}  \cdot \frac{c}{1}  = \frac{a \cdot c}{b \cdot 1} = \frac{a \cdot c}{b}  = \frac{a c}{b} 
\end{align*}

Division Bruch mit ganzer Zahl (Multiplikation mit Kehrwert):
\begin{align*}
\frac{a}{b} : c = \frac{a}{b} : \frac{c}{1} =  \frac{a}{b} \cdot \frac{1}{c} = \frac{a}{b \cdot c}  = \frac{a}{b c} 
\end{align*}

Division Bruch mit Bruch (Multiplikation mit Kehrwert):
\begin{align*}
\frac{a}{b} : \frac{c}{d} = \frac{a}{b}  \cdot \frac{d}{c} = \frac{a \cdot d }{b \cdot c} = \frac{a d }{b c} 
\end{align*}




\pagebreak
\subsection{Klammerrechnung}

Addition von Summanden (assoziativ):
\begin{align*}
 a + (b \pm c) = (a + b) \pm c = a + b \pm c 
\end{align*}
Subtraction von Summanden:
\begin{align*}
 a - (b \pm c) = a + (-1) \cdot  (b \pm c) = a + (  (-1) \cdot  b \pm  (-1) \cdot c)  =  a - b \mp c
\end{align*}


Multiplikation mit Summe:
\begin{align*}
 a (b \pm c) =  a \cdot (b \pm c) = a \cdot b \pm a \cdot c = ab \pm ac
\end{align*}
Bzw.
\begin{align*} 
 \sum_{i=1}^n ax_{i}  = a \cdot  \sum_{i=1}^n x_{i} = a  \sum_{i=1}^n x_{i}
 \end{align*}


Multiplikation von Summen:
\begin{align*}
 (a+b)(b+c) = a \cdot c +a \cdot d +b \cdot c +b \cdot d = ac +ad +bc +bd 
\end{align*}

Division einer Summe:
\begin{align*}
 (a \pm b):(c) = \frac{a \pm b}{c} = \frac{a}{c} \pm  \frac{b}{c} 
\end{align*}

\subsection{Binomische Formeln}
\begin{align*}
(a+b)^ 2 = a^ 2 + 2ab + b^2\\
(a-b)^ 2 =  a^ 2 - 2ab + b^2 \\
(a+b)(a-b) = a^ 2 - b^2
\end{align*}
 

\begin{align*} 
n! = 1\cdot2\cdot3...(n-1)n= \prod_{k = 1}^{n}k    (n \in \mathbb{N})
\end{align*}
 
 
$a^ 2 + b^2 $ (reell nicht zerlegbar!) 

\begin{align*} 
a^ 3 + b^3 = (a+b)( a^ 2 - ab + b^2) \\
a^ 3 - b^3 = (a-b)( a^ 2 + ab + b^2)
\end{align*}


\begin{tabular}{>{$n=}l<{$\hspace{12pt}}*{14}{c}}
0 &&&&&&&1&&&&&&& $(a+b)^ 0=1$\\
1 &&&&&&1&&1&&&&&& $(a+b)^ 1=1a+1b$ \\
2 &&&&&1&&2&&1&&&&&  $(a+b)^ 2=1a^ 2+2ab+1b^2$\\
3 &&&&1&&3&&3&&1&&&& $(a+b)^ 3=1a^3+3a^2 b+3ab^2+1b^3$\\
4 &&&1&&4&&6&&4&&1&&& $(a+b)^ 4=a^4+4a^3 b+6a^2b^2+4ab^3+b^4$\\
5 &&1&&5&&10&&10&&5&&1&&\\
6 &1&&6&&15&&20&&15&&6&&1&
\end{tabular}

Binominalkoeffizienten (n über k)
\begin{align*}
\binom{n}{k}= \frac{n(n-1)(n-2)...[n-(k-1)] }{k!} =  \frac{n!}{k!(n-k)!}
\end{align*}

\subsection{Potenzen}
Potenz eines Produktes gleicher Grundzahl: 
$a^m \cdot a^n = a^{m+n}$

Potenz einer Division/Bruches gleicher Grundzahl: $ \frac{a^m}{a^n} = a^{m-n}$ \\
$\longrightarrow$ $ a^0 =  \frac{a^n}{a^n} =   \frac{ \cancel{a^n} \cdot 1}{ \cancel{a^n} \cdot 1} = a^{n-n}  = 1$  ; $a \in \{a| a \neq 0\}$ \\
$\longrightarrow$ negative Exponenten: $ a^{-n} =  a^{0-n} =  \frac{a^0}{a^n}  =  \frac{1}{a^n} $ \\

Potenz eines Produktes mit gleichen Exponenten: $ {a^n}{b^n} = (ab)^{n}$ 

Potenz einer Division/Bruches mit gleichen Exponenten: $ \frac{a^n}{b^n} = (\frac{a}{b}) ^n$ 

Potenz einer Potenz: $ (a^m)^n = a^{m^n} = a^{m \cdot n}$ 

Potenz von 0: $ 0^n = 0 $ ;  $n \in\{n | n > 0\}$\\
Potenz 0 von 0: $ 0^0 = foobar $ 

\subsection{Wurzeln}

Radizieren eines Produktes $\sqrt[n]{ab} = \sqrt[n]{a} \cdot \sqrt[n]{b}$

Radizieren eines Quotienten $\sqrt[n]{\frac{a}{b} } = \frac{\sqrt[n]{a}}{\sqrt[n]{b}} $

Radizieren einer Potenz $\sqrt[n]{ a^m} = (\sqrt[n]{ a})^m = a^{\frac{m}{n}}$

Radizieren einer Wurzel $\sqrt[n]{\sqrt[m]{a}}  = \sqrt[n \cdot m]{a} = \sqrt[m]{\sqrt[n]{a}}$


\subsection{Logarithmen}
$$ b^x =n \longleftrightarrow x = \log_b n$$

Logarithmieren eines Produktes $\log_b (m \cdot n) = \log_b m + \log_b n $ 

Logarithmieren eines Quotienten $\log_b (\frac{m}{n}) = \log_b m - \log_b n  $ 

Logarithmieren einer Potenz $\log_b (m^n) =  n \cdot \log_b m $ 

Logarithmieren einer Wurzel $\log_b (\sqrt[n]{m}) =  \frac{1}{n} \cdot \log_b m $ 

Logarithmentsysteme \\
$\log_{10} n = \lg n $\\
$\log_{e} n = \ln n $


\section{Algebra}
\subsection{Gesetze}
Kommutativgesetz:Vertauschungsgesetz \\
Assoziativgesetz: Verknüpfungsgesetz oder auch Verbindungsgesetz (Klammern verschieben) \\
Distributivgesetz: Verteilungsgesetze (Ausklammern/Ausmultiplizieren)

\subsection{Ungleichungen}
\subsection{Gleichungen}
\subsection{Linieare Gleichungen}
\subsection{Determinanten}
\subsection{Quadratische Glechungen}
\subsection{Fundamentalsatz der Algebra}
Ein Polynom n-ten Grades lässt sich maximal in n Linearfaktoren zerlegen und hat maximal n Lösungen.

\section{Relationen und Funktionen}
\section{Folgen und Reihen}
\section{Differenzialrechnung}
\section{Integralrechnung}
\section{Geometrie}



\subsection{Planimetrie}
\subsection{Stereometrie}
\subsection{Trigonometrie}
\subsection{Geniometrie}
\section{Vectorrechung}
$\vec{B}$
\section{Matrixrechnung}

 
 $$
\Sigma=\left[
\begin{array}{ccc}
   \sigma_{11} & \cdots & \sigma_{1n} \\
   \vdots & \ddots & \vdots \\
   \sigma_{n1} & \cdots & \sigma_{nn}
\end{array}
\right]
$$

 $$
\Sigma=\left(
\begin{array}{ccc}
   \sigma_{11} & \cdots & \sigma_{1n} \\
   \vdots & \ddots & \vdots \\
   \sigma_{n1} & \cdots & \sigma_{nn}
\end{array}
\right)
$$


\newcommand*{\rowvec}[1]{\left( #1\right)}
\newcommand*{\rowvecVert}[1]{\left(\begin{array}{c}#1\end{array}\right)}

$$
\rowvec{1,2,3}
$$

$$
\left( 
\begin{array}{c}
1 \\ 2 \\ 3
\end{array}
\right)
$$
$$
\vec{B} = \rowvecVert{1 \\ 2 \\ 3}
$$

\section{Statistik}

\section{Konstanten}
\begin{align*} 
\pi =  4 \cdot \sum_{n=0}^\infty  \frac{(-1)^n}{2n+1} = \frac{1}{1} -  \frac{1}{3} +  \frac{1}{5} -  \frac{1}{7} + \frac{1}{9} \mp ... = \textbf{3,14159}...
\end{align*}
\begin{align*} 
e = 1 + \frac{1}{1}+\frac{1}{1 \cdot 2}+\frac{1}{1 \cdot 2  \cdot 3}+\frac{1}{1 \cdot 2  \cdot 3  \cdot 4}+... = \sum_{k=0}^\infty \frac{1}{k!} = \textbf{2,71828183}...
\end{align*}


\section{Zeichenerklärung}

\begin{tabular}[h]{l|l}
Zeichen &Bedeutung  \\
\hline
$\widehat{=}$ & enspricht  \\
$\longrightarrow$ & daraus folgt \\
$\sum$ & Summenzeichen \\
$\prod$ & Produktzeichen \\
$ \in$ & Element von \\
\end{tabular}





\end{document}
% ======================= END

$ \to $ % daraus folgt

$$\frac{a \pm b}{c} =  \frac{a}{c} \pm \frac{b}{c} $$

$\frac{a \pm b}{c} =  \frac{a}{c} \pm \frac{b}{c} $



