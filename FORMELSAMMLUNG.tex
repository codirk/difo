% https://www.grund-wissen.de/informatik/latex/mathematischer-formelsatz.html
% https://docplayer.org/28012320-Mathematische-formelsammlung.html
% add CC license
% use Command+Shift+Click in the PDF to jump to the source code line when using TeXShop

\author{Dirk~Messetat}
\date{\today}
\newcommand{\versionsnummer}{1.0.0-SNAPSHOT}

% URL:
\newcommand{\url}{}

% Titel:
\title{Mathematische \\
FORMELSAMMLUNG \\[2ex] \small Version: \versionsnummer \\[2ex] \texttt{ \url{http://github.com/codirk/formelsamlung} }\\
 \hfill\break
Klasse 5-13
} 

\documentclass[12pt,a4paper,fleqn,twoside,pdf,final]{scrartcl}
%\documentclass[12pt,a4paper,fleqn,twoside,pdf,final]{article}
%\documentclass{scrartcl}


\usepackage{graphicx}


\usepackage[ngerman]{babel}
\usepackage[utf8]{inputenc}
% \usepackage[T1]{fontenc}


\setlength{\parindent}{0in}
\setlength{\mathindent}{0pt}

\usepackage{amsfonts} 


\usepackage{helvet}
\usepackage{curves}
\usepackage{latexsym} 
\usepackage{textcomp} 

% damages  the pictures 
%\usepackage[dvips]{rotating}

\usepackage{geometry}

\geometry{left=0.5cm,textwidth=19cm,top=1.0cm,textheight=26cm}

\usepackage{array}

%\usepackage{creativecommons}
\usepackage[hidelinks]{hyperref}

\usepackage[
    type={CC},
    modifier={by-nc-sa},
    version={3.0},
]{doclicense}


% striked out text
\usepackage[normalem]{ulem}
\usepackage{cancel}

\usepackage{framed}
\usepackage{listings}
\usepackage{graphicx}
\usepackage{textcomp}
\usepackage{tikz}
\usepackage{tikz-qtree}
\usetikzlibrary{shapes,automata,arrows}
\usepackage{pgfplots}
\usepackage{color}




\usepackage{amsmath}
%\usepackage[draft]{graphics} % ohne Bilder (Entwurf)
\usepackage{graphics} % Bilder einbinden

\nonfrenchspacing
\renewcommand{\familydefault}{\sfdefault}


% newcommand
\newcommand{\stkout}[1]{\ifmmode\text{\cancel{\ensuremath{#1}}}\else\sout{#1}\fi}

\usepackage{float}
\usepackage{subfig}

\usetikzlibrary{calc,angles}
\usetikzlibrary{shapes,arrows,intersections}
\usetikzlibrary{matrix,fit,calc,trees,positioning,arrows,chains,shapes.geometric,shapes,angles}
\usetikzlibrary{quotes}

%\usetikzlibrary{calc,angles,quotes}
\tikzset{square/.style= { to path={ let \p1=(\tikztostart), \p2=(\tikztotarget),
  \p3=($(\p2)!1!90:(\p1)$), \p4=($(\p1)!1!-90:(\p2)$) in
  (\p2) foreach \i in {3,4,1,2} {--node[auto=right]{#1} (\p\i)}}}}

\usepackage{siunitx}



\begin{document}




\maketitle
\thispagestyle{empty}


% Fill with blanks to bottom of first page
\vfill
 

%\begin{flushright}
%    \copyright  2021 Dirk Messetat git [at] messetat [dot] com. \\ 
%    This work is licensed under a Creative Commons Attribution- ShareAlike 3.0 License.
%    To view a copy of this license visit:
%     \url{http://creativecommons.org/licenses/by-sa/3.0/legalcode}.
%\end{flushright}


\doclicenseThis


\pagebreak



% Inhaltsverzeichnis
\setcounter{tocdepth}{1}
\tableofcontents
\thispagestyle{empty}


\vfill

\begin{center}
\small{Dieses Dokument wurde mit \LaTeX{} gesetzt.}
\end{center}

\newpage

\clearpage
\pagenumbering{arabic} 

% https://de.wikipedia.org/wiki/Liste_mathematischer_Symbole

\section{Mengenlehre}

\subsection{Mengenverknüpfung}

\begin{tabular}[h]{lll}

Symbol &Menge & ggg \\
\hline
$\cup$ & Vereinigungsmenge	 & \{\} \\
$\cap$ & Schnittmenge	 & \{\} \\
$\setminus$ & Differenzmenge	 & \{\} \\
$\triangle$ & Symmetrische Differenz & \{\} \\
$\times$ & Kartesisches Produkt & \{\} \\
$\dot\cup$ & Vereinigung disjunkter Mengen  & \{\} \\
$\sqcup$ & Disjunkte Vereinigung der Mengen  & \{\} \\
$\bar{A}$ & Komplement & \{\} \\
$\mathcal{P}$ & Potenzmenge & \{\} \\
\end{tabular}

\subsection{Characteristische Mengen}

\begin{tabular}[h]{lll}
Symbol &Menge & ggg \\
\hline
$\emptyset$ & leere Menge & \{\} \\
$\mathbb{P}$ & Primzahlen & \{2,3,5,7,11,13,...\} \\
$\mathbb{N}$ & Natürliche Zahlen & \{1,2,3,...\} \\
$\mathbb{N}_0$ & Natürliche Zahlen mit 0 & \{0,1,2,3,..\} \\
$\mathbb{Z}$ & Ganze Zahlen & \{..,-3,-2,-1,0,1,2,3,...\} \\
$\mathbb{Z}^+$ & Ganze Zahlen & \{1,2,3,..\} \\
$\mathbb{Z}^-$ & Ganze Zahlen & \{...,-3,-2,-1\} \\
$\mathbb{Q}$ & Rationale Zahlen & \{ $\frac{p}{q} | (p \in \mathbb{Z}) \wedge (q \in  \mathbb{Z} \setminus \{0 \} ) $ \} \\
$\mathbb{R}$ & Reelle Zahlen & \{\} \\
$\mathbb{I}$ & Irrationale Zahlen & \{\} \\
$\mathbb{T}$ & Transzendente Zahlen & \{ $\pi, e$ \} \\
$\mathbb{C}$ & Komplexe Zahlen & \{\} \\
\end{tabular}


\subsection{Intervalle}

\begin{align*}
[a,b]  &=  \{x| a \leq x \leq b, x \in \mathbb{R}\} \\
(a,b) &= ]a,b[ =  \{x| a <  x < b, x \in \mathbb{R}\} \\
[a,b) &= [a,b[ =  \{x| a \leq  x < b, x \in \mathbb{R}\} \\
(a,b] &= ]a,b] =  \{x| a <  x \leq b, x \in \mathbb{R}\}
 \end{align*}
\pagebreak
\section{Arithmetik}
\subsection{Brüche}

Kürzen:
\begin{align*}
\frac{ac}{bc} = \frac{a \cdot c}{b \cdot c} =  \frac{a \cdot c : c }{b \cdot c : c} =  \frac{a \cdot 1 }{b \cdot 1 } =  \frac{a \cdot \stkout{c} }{b \cdot \stkout{c}}  =  \frac{a}{b}  
\end{align*}

Erweitern:
\begin{align*}
\frac{a}{b} =  \frac{ac}{bc} =   \frac{a \cdot c}{b \cdot c} 
\end{align*}

Addition und Subtraktion gleichnamiger Brüche:

\begin{align*}
  \frac{a}{c} \pm \frac{b}{c} =  \frac{a \pm b}{c} 
\end{align*}

Addition und Subtraktion nicht gleichnamiger Brüche:

\begin{flalign*}
\frac{a}{b} \pm \frac{c}{d} &= \frac{ad}{bd} \pm \frac{cb}{db} = \frac{ad \pm bc}{db} \\
\Rightarrow \frac{a}{b} \pm c &= \frac{a}{b} \pm \frac{c}{1} = \frac{a \pm bc}{b} 
\end{flalign*}


Multiplikation Bruch:
\begin{align*}
\frac{a}{b} c = c \frac{a}{b} = \frac{a}{b} \cdot c = \frac{a}{b}  \cdot \frac{c}{1}  = \frac{a \cdot c}{b \cdot 1} = \frac{a \cdot c}{b}  = \frac{a c}{b} 
\end{align*}

Achtung! : Addition bei Zahlen
\begin{align*}
2\frac{1}{3} = 2+\frac{1}{3} =  \frac{6}{3} + \frac{1}{3} = \frac{7}{3} 
\end{align*}



Division Bruch mit Bruch (Multiplikation mit Kehrwert):
\begin{flalign*}
\frac{a}{b} : \frac{c}{d}&= \frac{a}{b} \cdot \frac{d}{c}= \frac{a \cdot d }{b \cdot c}= \frac{a d }{b c} \\
\Rightarrow \frac{a}{b}:c&= \frac{a}{b} : \frac{c}{1}=\frac{a}{b} \cdot \frac{1}{c}=\frac{a}{b \cdot c}  = \frac{a}{b c} 
\end{flalign*}


\pagebreak
\subsection{Klammerrechnung}

Addition von Summanden (assoziativ):
\begin{align*}
 a + (b \pm c) = (a + b) \pm c = a + b \pm c 
\end{align*}
Subtraction von Summanden:
\begin{align*}
 a - (b \pm c) = a + (-1) \cdot  (b \pm c) = a + (  (-1) \cdot  b \pm  (-1) \cdot c)  =  a - b \mp c
\end{align*}


Multiplikation mit Summe:
\begin{align*}
 a (b \pm c) =  a \cdot (b \pm c) = a \cdot b \pm a \cdot c = ab \pm ac
\end{align*}
Bzw.
\begin{align*} 
 \sum_{i=1}^n ax_{i}  = a \cdot  \sum_{i=1}^n x_{i} = a  \sum_{i=1}^n x_{i}
 \end{align*}


Multiplikation von Summen:
\begin{align*}
 (a+b)(b+c) = a \cdot c +a \cdot d +b \cdot c +b \cdot d = ac +ad +bc +bd 
\end{align*}

Division einer Summe:
\begin{align*}
 (a \pm b):(c) = \frac{a \pm b}{c} = \frac{a}{c} \pm  \frac{b}{c} 
\end{align*}

\subsection{Binomische Formeln}
\begin{align*}
(a+b)^ 2 = a^ 2 + 2ab + b^2\\
(a-b)^ 2 =  a^ 2 - 2ab + b^2 \\
(a+b)(a-b) = a^ 2 - b^2
\end{align*}

$a^ 2 + b^2 $ (reell nicht zerlegbar!)


\begin{align*}
a^ 3 + b^3 = (a+b)( a^ 2 - ab + b^2) \\
a^ 3 - b^3 = (a-b)( a^ 2 + ab + b^2)
\end{align*}


 




Binominalkoeffizienten (n über k)
\begin{align*}
\binom{n}{k}= \frac{n(n-1)(n-2)...[n-(k-1)] }{k!} =  \frac{n!}{k!(n-k)!}
\end{align*}


\begin{align*}
n! = 1\cdot2\cdot3...(n-1)n= \prod_{k = 1}^{n}k    (n \in \mathbb{N})
\end{align*}


Pascalsches Dreieck

\begin{tabular}{>{$n=}l<{$\hspace{12pt}}*{14}{c}}
0 &&&&&&&1&&&&&&& $(a+b)^ 0=1$\\
1 &&&&&&1&&1&&&&&& $(a+b)^ 1=1a+1b$ \\
2 &&&&&1&&2&&1&&&&&  $(a+b)^ 2=1a^ 2+2ab+1b^2$\\
3 &&&&1&&3&&3&&1&&&& $(a+b)^ 3=1a^3+3a^2 b+3ab^2+1b^3$\\
4 &&&1&&4&&6&&4&&1&&& $(a+b)^ 4=a^4+4a^3 b+6a^2b^2+4ab^3+b^4$\\
5 &&1&&5&&10&&10&&5&&1&& ... \\
6 &1&&6&&15&&20&&15&&6&&1&
\end{tabular}





\subsection{Potenzen}
Potenz eines Produktes gleicher Grundzahl: 
$a^m \cdot a^n = a^{m+n}$

Potenz einer Division/Bruches gleicher Grundzahl: $ \frac{a^m}{a^n} = a^{m-n}$ \\
$\longrightarrow$ $ a^0 =  \frac{a^n}{a^n} =   \frac{ \cancel{a^n} \cdot 1}{ \cancel{a^n} \cdot 1} = a^{n-n}  = 1$  ; $a \in \{a| a \neq 0\}$ \\
$\longrightarrow$ negative Exponenten: $ a^{-n} =  a^{0-n} =  \frac{a^0}{a^n}  =  \frac{1}{a^n} $ \\

Potenz eines Produktes mit gleichen Exponenten: $ {a^n}{b^n} = (ab)^{n}$ 

Potenz einer Division/Bruches mit gleichen Exponenten: $ \frac{a^n}{b^n} = (\frac{a}{b}) ^n$ 

Potenz einer Potenz: $ (a^m)^n = a^{m^n} = a^{m \cdot n}$ 

Potenz von 0: $ 0^n = 0 $ ;  $n \in\{n | n > 0\}$\\
Potenz 0 von 0: $ 0^0 = foobar $ 

\subsection{Wurzeln}

Radizieren eines Produktes $\sqrt[n]{ab} = \sqrt[n]{a} \cdot \sqrt[n]{b}$

Radizieren eines Quotienten $\sqrt[n]{\frac{a}{b} } = \frac{\sqrt[n]{a}}{\sqrt[n]{b}} $

Radizieren einer Potenz $\sqrt[n]{ a^m} = (\sqrt[n]{ a})^m = a^{\frac{m}{n}}$

Radizieren einer Wurzel $\sqrt[n]{\sqrt[m]{a}}  = \sqrt[n \cdot m]{a} = \sqrt[m]{\sqrt[n]{a}}$


\subsection{Logarithmen}
$$ b^x =n \longleftrightarrow x = \log_b n$$

Logarithmieren eines Produktes $\log_b (m \cdot n) = \log_b m + \log_b n $ 

Logarithmieren eines Quotienten $\log_b (\frac{m}{n}) = \log_b m - \log_b n  $ 

Logarithmieren einer Potenz $\log_b (m^n) =  n \cdot \log_b m $ 

Logarithmieren einer Wurzel $\log_b (\sqrt[n]{m}) =  \frac{1}{n} \cdot \log_b m $ 

Logarithmentsysteme \\
$\log_{10} n = \lg n $\\
$\log_{e} n = \ln n $


\section{Algebra}
\subsection{Gesetze}
Kommutativgesetz:Vertauschungsgesetz \\
Assoziativgesetz: Verknüpfungsgesetz oder auch Verbindungsgesetz (Klammern verschieben) \\
Distributivgesetz: Verteilungsgesetze (Ausklammern/Ausmultiplizieren)

\subsection{Ungleichungen}
\subsection{Gleichungen}
\subsection{Linieare Gleichungen}
\subsection{Determinanten}
\subsection{Quadratische Gleichungen}
Allgemeine Form:
$ ax^2 + bx + c = 0 $

Normalform:
$ x^2 + px + q = 0 $ mit $p=\frac{b}{a}$ und $q=\frac{c}{a}$

Lösung: $x_{1,2}= - \frac{p}{2} \pm \sqrt[]{( \frac{p}{2})^2 -q }$

Linearfactoren (Produktform quadratischer Terme): $(x-x_1)\cdot (x-x_2 )=0$


\subsection{Fundamentalsatz der Algebra}
Ein Polynom n-ten Grades lässt sich maximal in n Linearfaktoren zerlegen und hat maximal n Lösungen.

\pagebreak
\section{Relationen und Funktionen}
TDB

\section{Folgen und Reihen}
\subsection{Arithmetische folgen und Reihen}

\begin{align*}
d &=a_{n+1} - a_n = const. ; n \in \mathbb{N}
\end{align*}

Arithmetisches Mittel
\begin{align*}
    a_n=\frac{a_{n-1}+a_{n+1}}{2}
\end{align*}

Funktionsforschrift
\begin{align*}
    a_n = a_1+(n-1) \cdot d ; d \in \{d|d \neq 0\}
\end{align*}

Summenformel
\begin{align*}
    \sum_{n=1}^{n} a_n &= \frac{a_1+a_n}{2} \cdot n \\
    &= \sum_{n=1}^{n} a_1 +(n-1) \cdot d \\
    &= \frac{n}{2}[2a_1+(n-1) \cdot d]
\end{align*}



\subsection{Geometrische folgen und Reihen}
\section{Differenzialrechnung}
\paragraph{Stetigkeit}
\begin{align*}
    y &= f(x) \textrm{ ist in } a \in\mathbb{R} \textrm{ stetig, wenn}\\
    & \lim_{x \to a} f(x) = f(a) \\
    & \textrm{ ist}
\end{align*}

\paragraph{Differenzierbarkeit}
\begin{align*}
    y &= f(x) \textrm{ ist in } a \in\mathbb{R} \textrm{ differenzierbar, wenn}\\
   & \lim_{x \to a} \frac{f(x)-f(a)}{x-a} \textrm{ oder } \lim_{\Delta x \to 0} \frac{f(x-\Delta x) -f (x)}{\Delta x}\\
   & \textrm{ existiert}
\end{align*}

\paragraph{Ableitungsregeln}
\begin{align*}
    Potenzregel &: y= x^n                   &| &y^{\prime} = n \cdot x^{n-1} \\
    Konstantenregel &: y= a \cdot x^n       &| &y^{\prime} = a \cdot n \cdot x^{n-1} \\
    Summenregel &: y= f(x) \pm g(x)         &| &y^{\prime} = f^{\prime}(x) \pm g^{\prime}(x) \\
    Produktregel &: y= u(x) \cdot v(x)      &| &y^{\prime} = u^{\prime}(x) \cdot v(x) + u(x) \cdot v^{\prime}(x)\\
    Quotientenregel &: y= \frac{u(x)}{v(x)} &| &y^{\prime} = \frac{u^{\prime}(x) \cdot v(x) - u(x) \cdot v^{\prime}(x)}{v(x)^2}\\
    Kettenregel &: y= f(u(x))               &| &y^{\prime} = u^{\prime}(x) \cdot u^{\prime}(x) = \frac{dy}{du} \cdot \frac{du}{dx} \\
\end{align*}

\paragraph{Transzendente Funktionen}
\begin{align*}
   y &= \mathrm{e}^x    &| &y^{\prime} = \mathrm{e}^x \\
   y &= \ln{x}          &| &y^{\prime} = \frac{1}{x} \\
   y &= \sin{x}         &| &y^{\prime} = \cos{x} \\
   y &= \cos{x}         &| &y^{\prime} = - \sin{x} \\
   y &= \tan{x}         &| &y^{\prime} = \frac{1}{\cos^2{x}} \\
   y &= \cot{x}         &| &y^{\prime} = \frac{-1}{\sin^2{x}} \\
\end{align*}

\paragraph{Extrempunkte}
\begin{align*}
    y^{\prime}=0 &\land y^{\prime\prime} > 0  \Rightarrow  \textrm{Maximum} \\
    y^{\prime}=0 &\land y^{\prime\prime} < 0  \Rightarrow  \textrm{Minimum}
\end{align*}


\paragraph{Wendepunkte}
\begin{align*}
    y^{\prime\prime}=0 &\land y^{\prime\prime\prime} \ne 0 \land y^{\prime} \ne 0
\end{align*}
\paragraph{Sattelpunkt}
\begin{align*}
    y^{\prime\prime}=0 &\land y^{\prime\prime\prime} \ne 0 \land y^{\prime} = 0
\end{align*}





\pagebreak

\section{Integralrechnung}

\subsection{Unbestimmtes Integral}

\begin{align*}
   \int f(x) \,dx = F(x) + C
\end{align*}

\subsection{Bestimmtes Integral}

\begin{align*}
   \int_{a}^{b} f(x) \,dx = F_{(b)} -F_{(a)} = A
\end{align*}


\paragraph{Potenzregel}
\begin{align*}
   \int x^n \,dx = \frac{x^{n+1}}{n+1} + C
\end{align*}
\paragraph{Summenregel}
\begin{align*}
   \int [u(x) \pm v(x)] \,dx = \int u(x) \,dx \pm \int v(x) \,dx
\end{align*}
\paragraph{Productregel}
\begin{align*}
   \int u(x) \cdot v^{\prime}(x) \,dx = u(x) \cdot v(x) - \int u^{\prime}(x) \cdot v(x) \,dx
\end{align*}


\section{Geometrie}
\subsection{Planimetrie}
\subsubsection{Strahlensätze}
%TODO picture
\paragraph{1.}
\begin{align*}
    \overline{SA_1} : \overline{SB_1} = \overline{A_1A_2} : \overline{B_1B_2} \\
    \overline{SA_1} : \overline{SA_1} = \overline{SB_1} : \overline{SB_2} \\
\end{align*}
Entsprechende Abschnitte auf den Strahlen stehen im gleichen Verhältnis

\paragraph{2.}
\begin{align*}
    \overline{CB} : \overline{C_1B_1}: \overline{C_2B_2}    &= \overline{SC} : \overline{SC_1} : \overline{SC_2}  \\
                                                            &= \overline{SB} : \overline{SB_1} : \overline{SB_2}  \\
\end{align*}
\begin{align*}
    \overline{AB} : \overline{A_1B_1}: \overline{A_2B_2}    &= \overline{SA} : \overline{SA_1} : \overline{SA_2}  \\
                                                            &= \overline{SB} : \overline{SB_1} : \overline{SB_2}  \\
\end{align*}
Entsprechende Abschnitte auf den Parallelen stehen im gleichen Verhältnis wie die zugehörigen, vom Scheitel aus zu messenden Abschnitte auf den Strahlen.

\paragraph{3.}
\begin{align*}
    \overline{AB} : \overline{BC} = \overline{A_1B_1} : \overline{B_1C_1} = \overline{A_2B_2} : \overline{B_2C_2} \\
\end{align*}
Entsprechende Abschnitte auf den Parallelen stehen im gleichen Verhältnis.

\subsubsection{Satz des Pythagoras}
%TODO picture
\begin{align*}
    a^2+b^2=c^2
\end{align*}

\subsubsection{Höhensatz des Euklid}
%TODO picture
\begin{align*}
    h^2=p \cdot q
\end{align*}


%\begin{figure}[htb]
%\begin{center}
%	\includegraphics[height=1in,width=1in,angle=-90]{sections/08_Geometrie/pythagoras.standalone.pdf}
%\caption{This is a figure.}
%\end{center}
%\end{figure}

%\begin{figure}
%    \centering
%    \subfloat{
%	\includegraphics[height=2in,width=2in,angle=-90]{sections/08_Geometrie/pythagoras.standalone.pdf} 
%   }\hfill
%   \subfloat{
%      \begin{minipage}[t]{120mm}
%   	$A = \frac{d^2 \cdot \pi}{4}$ \\
%	$U = d \cdot \pi  $
%      \end{minipage}
%    }
%  \caption{Abbildungsname}
%  \label{fig:decision-tree}
%\end{figure}

\subsubsection{Kathetensatz}
%TODO picture
\begin{align*}
    a^2=c \cdot p \\
    b^2=c \cdot q \\
\end{align*}

\subsubsection{Trapez}
%TODO picture
\begin{align*}
\end{align*}

\subsubsection{Kreis}
%TODO picture
\begin{align*}
    A &= \frac{d^2 \cdot \pi}{4} \\
    U &= d \cdot \pi \\
\end{align*}

\subsubsection{Kreisausschnitt}
%TODO picture
\begin{align*}
    A &= \frac{d^2 \cdot \pi}{4} \cdot \frac{\alpha^\circ}{360^\circ} = \frac{b \cdot r}{2} \\
    b &= \frac{d \cdot \pi \cdot \alpha^\circ }{ 360^\circ} \\
\end{align*}

\subsubsection{Ellipse}
%TODO picture
\begin{align*}
    A &= \frac{D \cdot d \cdot \pi}{4} = a \cdot b \cdot \pi \\
    U &\approx \frac{D+d}{2} \cdot \pi \\
\end{align*}


\subsection{Stereometrie}
\paragraph{Gerade Körper}
\begin{align*}
    V = A \cdot h
\end{align*}

\paragraph{Spitze Körper}
\begin{align*}
    V = \frac{1}{3} \cdot A \cdot h
\end{align*}

\paragraph{Stumpfe Körper}
\begin{align*}
    V = \frac{1}{3} \cdot h ( A_1 + \sqrt{A_1 \cdot A_2} + A_2)
\end{align*}

\paragraph{Kugel}
\begin{align*}
    V &= \frac{4}{3} \cdot \pi \cdot r^3 \\
    O &= 4 \cdot \pi \cdot r^2 \\
\end{align*}

\subsection{Trigonometrie}
\subsection{Geniometrie}
\section{Vectorrechung}

\newcommand{\myvec}[3]{
\left(
    \begin{array}{c}
    #1 \\ #2 \\ #3
    \end{array}
\right)
}


\subsection{Darstellung}
$$
\vec{a}= \vec{a}_x +\vec{a}_y +\vec{a}_z = a_xi +a_yj + a_zk= \myvec{a_x}{a_y}{a_z}
$$

\subsection{Betrag}
$$
|\vec{a}|= a = \sqrt{a_x^2 +a_y^2 + a_z^2}
$$

\subsection{Einheitsvector}
$$
  \vec{a}_0 = \frac{\vec{a}}{|\vec{a}|}  \longrightarrow \vec{a}_0 \cdot \vec{a} = a \cdot \vec{a}
$$

\subsection{Addition}
\begin{align*}
  \vec{a} \pm \vec{b} &= (a_xi +a_yj + a_zk) \pm (b_xi +b_yj + b_zk)\\
  &= (a_x+b_x)i +(a_y+b_y)j + (a_z+b_z)k \\
  &=  \myvec{a_x \pm b_x}{a_y \pm b_y}{a_z \pm b_z}
\end{align*}

\subsection{Nullvector}
\begin{align*}
  \vec{a}-\vec{a}&= \vec{0} \\
  |\vec{0}|&=0
\end{align*}

\subsection{Multiplikation mit Skalar}
\begin{align*}
 n \cdot \vec{a} =   \myvec{n \cdot a_x}{n \cdot a_y}{n \cdot a_z}
\end{align*}

\subsection{Multiplikation von Vectoren (Skalarprodukt)}
\begin{align*}
 \vec{a} \cdot \vec{b} =  |\vec{a}| \cdot |\vec{b}| \cos{(\vec{a},\vec{b})}
\end{align*}


\subsection{Gesetze}
\begin{align*}
 \vec{a} \cdot \vec{b} = \vec{b} \cdot \vec{a} (assoziativ) \\
 n \cdot (\vec{a} \cdot \vec{b}) =  (n \cdot \vec{a}) \cdot \vec{b} = \vec{a} \cdot(n \cdot \vec{b})  (kommutativ) \\
 \vec{a} \cdot (\vec{b} \pm \vec{b}) = \vec{a} \cdot \vec{b} \pm  \vec{a} \cdot \vec{c} (distributiv) \\
\end{align*}

\subsection{Sonderfall}
\begin{align*}
 \vec{a} \cdot \vec{a} &= |\vec{a}| \cdot |\vec{a}| \cdot \cos 0^\circ \\
 &=  \vec{a}^2 = a^2
\end{align*}

\subsection{Kreuzprodukt}
\begin{align*}
 \vec{a} \times \vec{b} &= \left|
                               \begin{array}{ccc}
                                  i & j & k \\
                                  a_x & a_y & a_z \\
                                  b_x & b_y & b_z \\
                               \end{array}
                               \right|\\
                         =   \left|
                              \begin{array}{ccc}
                                 a_y & a_z  \\
                                 b_y & b_z \\
                              \end{array}
                              \right|i + \\
                              \left|
                              \begin{array}{ccc}
                                 a_z & a_x  \\
                                 b_z & b_x \\
                              \end{array}
                              \right|j + \\
                             \left|
                             \begin{array}{ccc}
                                a_x & a_y  \\
                                b_x & b_y \\
                             \end{array}
                             \right|k
\end{align*}

\subsection{Spatprodukt}
\begin{align*}
 \vec{a} \cdot \vec{b} \cdot \vec{c} &=    \left|
                                           \begin{array}{ccc}
                                              a_x & a_y & a_z \\
                                              b_x & b_y & b_z \\
                                              c_x & c_y & c_z \\
                                           \end{array}
                                           \right|\\
\end{align*}






\section{Matrixrechnung}
 $$
\Sigma=\left[
\begin{array}{ccc}
   \sigma_{11} & \cdots & \sigma_{1n} \\
   \vdots & \ddots & \vdots \\
   \sigma_{n1} & \cdots & \sigma_{nn}
\end{array}
\right]
$$

 $$
\Sigma=\left(
\begin{array}{ccc}
   \sigma_{11} & \cdots & \sigma_{1n} \\
   \vdots & \ddots & \vdots \\
   \sigma_{n1} & \cdots & \sigma_{nn}
\end{array}
\right)
$$


\newcommand*{\rowvec}[1]{\left( #1\right)}
\newcommand*{\rowvecVert}[1]{\left(\begin{array}{c}#1\end{array}\right)}

$$
\rowvec{1,2,3}
$$

$$
\left(
\begin{array}{c}
1 \\ 2 \\ 3
\end{array}
\right)
$$
$$
\vec{B} = \rowvecVert{1 \\ 2 \\ 3}
$$
\section{Stochastik}
\subsection{Wahrscheinlichkeit}
TBD
\subsection{Statistik}
TBD


\section{Konstanten}
\begin{align*} 
\pi =  4 \cdot \sum_{n=0}^\infty  \frac{(-1)^n}{2n+1} = \frac{1}{1} -  \frac{1}{3} +  \frac{1}{5} -  \frac{1}{7} + \frac{1}{9} \mp ... = \textbf{3,14159}...
\end{align*}
\begin{align*} 
e = 1 + \frac{1}{1}+\frac{1}{1 \cdot 2}+\frac{1}{1 \cdot 2  \cdot 3}+\frac{1}{1 \cdot 2  \cdot 3  \cdot 4}+... = \sum_{k=0}^\infty \frac{1}{k!} = \textbf{2,71828183}...
\end{align*}


\section{Zeichenerklärung}

\begin{tabular}[h]{l|l}
Zeichen &Bedeutung  \\
\hline
$\widehat{=}$ & enspricht  \\
$\longrightarrow$ & daraus folgt \\
$\sum$ & Summenzeichen \\
$\prod$ & Produktzeichen \\
$ \in$ & Element von \\
\end{tabular}





\end{document}
% ======================= END

$ \to $ % daraus folgt

$$\frac{a \pm b}{c} =  \frac{a}{c} \pm \frac{b}{c} $$

$\frac{a \pm b}{c} =  \frac{a}{c} \pm \frac{b}{c} $



