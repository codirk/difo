\documentclass{scrartcl}


\usepackage[ngerman]{babel}
\usepackage[utf8]{inputenc}
\usepackage[T1]{fontenc}


\usepackage{framed}
\usepackage{listings}
\usepackage{graphicx}
\usepackage{textcomp}
\usepackage{amsmath}

\usepackage{tikz}
\usepackage{tikz-qtree}
\usetikzlibrary{shapes,automata,arrows}
\usepackage{pgfplots}
\usepackage{color}



\begin{document}

%---

\subsection{Picture}


{
\setlength{\unitlength}{1cm}
\begin{picture}(2,2)
    \put(0,0){\vector(4,0){3.3}}
    \put(0,0){\vector(0,4){3.3}}
    \put(3.4,-0.1){$x$}
    \put(-0.1,3.4){$y$}    
    \multiput(0,0)(1,0){4}{\line(0,-1){0.1}}
    \multiput(0,0)(0,1){4}{\line(-1,0){0.1}}

    \put(0,2){\vector(1,-1){2}}
    \put(2,0){\vector(1,1){1}}
    \put(2,2){\circle*{1}}
    \put(2,2){\circle{2}}    
\end{picture}
}


%----

\subsection{Tikz}


\begin{tikzpicture}
\draw[->] (0, 0) -- (0, 4);
\draw[->] (0, 0) -- (4, 0);
\node at (0, 4.2) {$y$}; 
 \node at (4.125, -0.25) {$x$};
 \foreach \x in {0, 1, 2, 3}
 {
  \draw (\x,0) -- (\x,-0.2);
  \node at (\x,-0.75) {$\x$}; 
  \draw (0, \x) -- (-0.2, \x);
  \node at (-0.5, \x-0.25) {$\x$};
 }
\coordinate (m) at (2,2);
\draw (m) circle (0.5);
\draw[fill] (m) circle (0.25);
\draw[->] (0, 2) -- (2, 0);
\draw[->] (2, 0) -- (3, 1);
\end{tikzpicture}


%---

\subsection{Automata}

%\usepackage{tikz}
%\usetikzlibrary{automata, arrows}

Graph des regul\"aren Ausdrucks \texttt{(ab*|b|ca)}

\begin{tikzpicture}[shorten >=1pt, 
    node distance=2.2cm, >=stealth, thick, auto, bend angle=45]
  \tikzstyle{every state}=[draw=blue!50, very thick, fill=blue!20]

  \node[state, initial]      (A)                                    {$S_0$};
  \node[state, accepting]    (B) [below of=A]                       {$S_1$};
  \node[state, accepting]    (C) [right of=B, node distance=1.25cm] {$S_2$};
  \node[state]               (D) [below of=B]                       {$S_3$};

  \path[->] (A) edge                             node {$a$} (B)
            (B) edge [loop below]                node {$b$} ()
            (A) edge                             node {$b$} (C)
            (A) edge [bend right, bend angle=45] node {$c$} (D)
            (D) edge [bend right]                node {$a$} (C);
\end{tikzpicture}



%\usepackage{tikz}
%\usetikzlibrary{shapes,arrows}

\subsection{Automata 2}

\tikzstyle{block} = [rectangle, draw, 
    text width=8em, text centered, rounded corners, minimum height=4em]
	\tikzstyle{line} = [draw, thick, -> ] %, -latex']
 	\tikzstyle{cloud} = [ellipse, draw, node distance=4cm,
     minimum height=2em]

	\begin{tikzpicture}[node distance = 2cm, auto]
    % Place nodes
    	\node [cloud] (inhalt) {\small \texttt{inhalt.tex}};
     	\node [block, left of=inhalt, above of=inhalt] (handout) 
            {\small \texttt{handout.tex}};
    	\node [block, right of=inhalt, above of=inhalt] (pres) 
            {\small \texttt{presentation.tex}};
    	\path [line] (inhalt) -- (pres);
    	\path [line] (inhalt) -- (handout);
	\end{tikzpicture}
    
%---


\subsection{trees}


\begin{tikzpicture}
\Tree
[.S 
    [.NP [.Det the ] [.N cat ] ]
    [.VP [.V sat ]
        [.PP 
            [.P on ]
            [.NP [.Det the ] [.N mat ] ] 
        ]
    ] 
]
\end{tikzpicture}

%---


\subsection{Pythagoras}

\begin{tikzpicture}
    % for a, A(1, 1.5) top right, side length = 1
    \fill [red!15]   (0,0) rectangle (1,-1);     
    % for b, B(0, 0) bottom left, side length = 1.5
    \fill [green!15] (1,0) rectangle (2.5,1.5);

    % length of c, C(1, 0) bottom right
    \pgfmathsetmacro{\lengthC}{(sqrt(1^2+1.5^2))}
    % angle between a and c
    \pgfmathsetmacro{\myAngle}
        {90 - acos((\lengthC^2 + 1^2 - 1.5^2) / (2*1.8*1))}
    \begin{scope}[rotate=-\myAngle] 
        \fill [blue!15] (0,0) rectangle (-\lengthC,\lengthC); % for c
    \end{scope}

    % labels
    \draw [gray,decorate,decoration={brace,amplitude=5pt}]
          (1,0) -- (0,0)
    node [black,midway,below=3pt] {\footnotesize $a$};
    \draw [gray,decorate,decoration={brace,amplitude=5pt}]
          (1,1.5) -- (1,0)
    node [black,midway,right=3pt] {\footnotesize $b$};
    \draw [gray,decorate,decoration={brace,amplitude=5pt}]
          (0,0) -- (1,1.5)
    node [black,midway,left=1.5pt,yshift=6pt] {\footnotesize $c$};
\end{tikzpicture}



\subsection{Trigonometrie}

\begin{tikzpicture}[scale=2,cap=round]
  % Local definitions
  \def\costhirty{0.8660256}
  % Colors
  \colorlet{anglecolor}{orange}
  \colorlet{sincolor}{red}
  \colorlet{tancolor}{green!80!black}
  \colorlet{coscolor}{blue}
  % Styles
  \tikzstyle{axes}=[]
  \tikzstyle{important line}=[very thick]
  \tikzstyle{information text}=[]  
  % The graphic
  \draw[style=help lines,step=0.5cm] (-1.4,-1.4) grid (1.4,1.4);
  \draw (0,0) circle (1cm);
  \begin{scope}[style=axes]
    \draw[->] (-1.5,0) -- (1.5,0) node[right] {$x$} coordinate(x axis);
    \draw[->] (0,-1.5) -- (0,1.5) node[above] {$y$} coordinate(y axis);
    \foreach \x/\xtext in {-1, 1}
       \draw[xshift=\x cm] (0pt,1pt) -- (0pt,-1pt) node[below,fill=white] {$\xtext$};
    \foreach \y/\ytext in {-1, 1}
       \draw[yshift=\y cm] (1pt,0pt) -- (-1pt,0pt) node[left,fill=white] {$\ytext$};
  \end{scope}
  \filldraw[fill=orange!20,draw=anglecolor] (0,0) -- (3mm,0pt) arc(0:30:3mm);
  \draw (15:2mm) node[anglecolor] {$\alpha$};
  \draw[style=important line,sincolor]
    (30:1cm) -- node[left=1pt,fill=white] {$\sin \alpha$} (30:1cm |- x axis);
  \draw[style=important line,coscolor]
    (30:1cm |- x axis) -- node[below=2pt,fill=white] {$\cos \alpha$} (0,0);
  \draw[style=important line,tancolor] (1,0) -- node[right=1pt,fill=white] {
    $\displaystyle \tan \alpha \color{black}=
    \frac{{\color{sincolor}\sin \alpha}}{\color{coscolor}\cos \alpha}$}
    (intersection of 0,0--30:1cm and 1,0--1,1) coordinate (t);
  \draw (0,0) -- (t);
  \draw[xshift=3.2cm]
    node[text width=3cm]
     {\small Seitenverhältnisse im rechtwinkligen Dreieck:
      \[\sin \alpha = \frac{\text{Gegenkathete}}{\text{Hypothenuse}}\]
      \[\cos \alpha = \frac{\text{Ankathete}}{\text{Hypothenuse}}\]
      \[\tan \alpha = \frac{\text{Gegenkathete}}{\text{Ankathete}}\]};
\end{tikzpicture}


\subsection{PGFPlots}

\begin{tikzpicture}
    \begin{axis}[width=190pt,axis x line=middle, axis y line=center, tick align=outside]
        \addplot+[mark=none, smooth] (\x,{1/10*\x^2});
		\addplot+[mark=none, smooth] (\x,{sin(\x r)});
    \end{axis}
\end{tikzpicture}





\end{document}
