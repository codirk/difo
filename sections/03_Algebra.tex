\section{Algebra}
\subsection{Gesetze}
Kommutativgesetz:Vertauschungsgesetz \\
Assoziativgesetz: Verknüpfungsgesetz oder auch Verbindungsgesetz (Klammern verschieben) \\
Distributivgesetz: Verteilungsgesetze (Ausklammern/Ausmultiplizieren)






% \subsection{Ungleichungen}

\subsection{Gleichungen}
Äquivalenzumformungen (T $\widehat{=}$ Term)
\begin{align*}
T_1 &= T_2 &\longrightarrow& & T_2 &= T_1 \\
T_1 &= T_2 | \pm T_3 &\longrightarrow& & T_1 \pm T_3 &= T_2 \pm T_3 \\
T_1 &= T_2 | \cdot a &\longrightarrow& & T_1 \cdot a &= T_2 \cdot a &a \in \{a|a \neq 0\} \\
T_1 &= T_2 | : a &\longrightarrow& & \frac{T_2}{a} &= \frac{T_2}{a} &a \in \{a|a \neq 0\} \\
\end{align*}


\subsubsection{Linieare Gleichungen}

\paragraph{2 Gleichungen mit 2 Unbekannten}

\begin{align*}
a x + by - c = 0\\
d x + e y - f = 0\\
oder \\
a_{11} x_1 + a_{12} x_2 - K_1 = 0\\
a_{21} x_1 + a_{22} x_2 - K_2 = 0\\
\end{align*}

\subparagraph{Determinaten}
\begin{align*}
D= a_{11} a_{22} - a_{12} a_{21} = \left|
    \begin{array}{ccc}
       a_{11} & a_{12} \\
       a_{21} & a_{22} \\
    \end{array}
    \right|
\end{align*}

\begin{align*}
D_{{x}_1}= K_1 a_{22} - K_2 a_{12} = \left|
     \begin{array}{ccc}
        K_1 & a_{12} \\
        K_2 & a_{22} \\
     \end{array}
     \right|
\end{align*}

\begin{align*}
D_{{x}_2}= K_2 a_{11} - K_1 a_{21} = \left|
     \begin{array}{ccc}
        a_{11} & K_1 \\
        a_{21} & K_2 \\
     \end{array}
     \right|
\end{align*}

\subparagraph{Cramer'sche Regel}
Wenn alle Gleichungen linear unabhängig sind
\begin{align*}
D &\neq 0
\end{align*}
dann
\begin{align*}
x_1 &= \frac{ D_{{x}_1} }{D} \\
x_2 &= \frac{ D_{{x}_2} }{D} \\
\end{align*}



\paragraph{m-Gleichungen mit n-Unbekannten}
Die Cramer'sche Regel gild analog für m Gleichungen mit n Variablen.
\begin{align*}
a_{11}x_{1}+a_{12}x_{2}+a_{13}x_{1}+...+a_{1n}x_{2}-c_{1}&=0 \\
a_{21}x_{1}+a_{22}x_{2}+a_{23}x_{1}+...+a_{2n}x_{2}-c_{1}&=0 \\
a_{31}x_{1}+a_{32}x_{2}+a_{33}x_{1}+...+a_{3n}x_{2}-c_{1}&=0 \\
...\\
a_{m1}x_{1}+a_{m2}x_{2}+a_{m3}x_{1}+...+a_{mn}x_{2}-c_{m}&=0 \\
\end{align*}

\subparagraph{Determinaten}
$$
D=\left|
\begin{array}{ccc}
   a_{11} & \cdots & a_{1n} \\
   \vdots & \ddots & \vdots \\
   a_{n1} & \cdots & a_{nn}
\end{array}
\right|
$$



\subsection{Quadratische Gleichungen}
Allgemeine Form:
$ ax^2 + bx + c = 0 $

Normalform:
$ x^2 + px + q = 0 $ mit $p=\frac{b}{a}$ und $q=\frac{c}{a}$

Lösung: $x_{1,2}= - \frac{p}{2} \pm \sqrt[]{( \frac{p}{2})^2 -q }$

Linearfactoren (Produktform quadratischer Terme): $(x-x_1)\cdot (x-x_2 )=0$

\subsection{Polynomfunktion n-ten Grades}
Ein Polynom n-ten Grades lässt sich maximal in n Linearfaktoren zerlegen und hat maximal n Lösungen bzw. hat maximal n Nullstellen.
\begin{align*}
f(x) &= a_nx^n+a_{n-1}x^{n-1}+...+a_1x^1+a_0 \\
p(x) &=a_n(x-n_1)(x-n_2)(x-n_3)...(x-n_N) \textrm{ mit jeweils den Nullstellen } n_k
\end{align*}
