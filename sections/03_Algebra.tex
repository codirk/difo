\section{Algebra}
\subsection{Gesetze}
Kommutativgesetz:Vertauschungsgesetz \\
Assoziativgesetz: Verknüpfungsgesetz oder auch Verbindungsgesetz (Klammern verschieben) \\
Distributivgesetz: Verteilungsgesetze (Ausklammern/Ausmultiplizieren)






% \subsection{Ungleichungen}

\subsection{Gleichungen}
Äquivalenzumformungen (T $\widehat{=}$ Term)
\begin{align*}
T_1 &= T_2 &\longrightarrow& & T_2 &= T_1 \\
T_1 &= T_2 | \pm T_3 &\longrightarrow& & T_1 \pm T_3 &= T_2 \pm T_3 \\
T_1 &= T_2 | \cdot a &\longrightarrow& & T_1 \cdot a &= T_2 \cdot a &a \in \{a|a \neq 0\} \\
T_1 &= T_2 | : a &\longrightarrow& & \frac{T_2}{a} &= \frac{T_2}{a} &a \in \{a|a \neq 0\} \\
\end{align*}


\subsubsection{Linieare Gleichungen}
\paragraph{Determinanten}
\paragraph{Cramer'sche Regel}

\subsection{Quadratische Gleichungen}
Allgemeine Form:
$ ax^2 + bx + c = 0 $

Normalform:
$ x^2 + px + q = 0 $ mit $p=\frac{b}{a}$ und $q=\frac{c}{a}$

Lösung: $x_{1,2}= - \frac{p}{2} \pm \sqrt[]{( \frac{p}{2})^2 -q }$

Linearfactoren (Produktform quadratischer Terme): $(x-x_1)\cdot (x-x_2 )=0$

\subsection{Polynomfunktion n-ten Grades}
Ein Polynom n-ten Grades lässt sich maximal in n Linearfaktoren zerlegen und hat maximal n Lösungen bzw. hat maximal n Nullstellen.
\begin{align*}
f(x) &= a_nx^n+a_{n-1}x^{n-1}+...+a_1x^1+a_0 \\
p(x) &=a_n(x-n_1)(x-n_2)(x-n_3)...(x-n_N) \textrm{ mit jeweils den Nullstellen } n_k
\end{align*}
