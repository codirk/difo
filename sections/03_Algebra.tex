\section{Algebra}
\subsection{Gesetze}
Kommutativgesetz:Vertauschungsgesetz \\
Assoziativgesetz: Verknüpfungsgesetz oder auch Verbindungsgesetz (Klammern verschieben) \\
Distributivgesetz: Verteilungsgesetze (Ausklammern/Ausmultiplizieren)

\subsection{Ungleichungen}
\subsection{Gleichungen}
\subsection{Linieare Gleichungen}
\subsection{Determinanten}
\subsection{Quadratische Gleichungen}
Allgemeine Form:
$ ax^2 + bx + c = 0 $

Normalform:
$ x^2 + px + q = 0 $ mit $p=\frac{b}{a}$ und $q=\frac{c}{a}$

Lösung: $x_{1,2}= - \frac{p}{2} \pm \sqrt[]{( \frac{p}{2})^2 -q }$

Linearfactoren (Produktform quadratischer Terme): $(x-x_1)\cdot (x-x_2 )=0$


\subsection{Fundamentalsatz der Algebra}
Ein Polynom n-ten Grades lässt sich maximal in n Linearfaktoren zerlegen und hat maximal n Lösungen.
