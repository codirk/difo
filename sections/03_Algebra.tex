\section{Algebra}
\subsection{Gesetze}
Kommutativgesetz:Vertauschungsgesetz \\
Assoziativgesetz: Verknüpfungsgesetz oder auch Verbindungsgesetz (Klammern verschieben) \\
Distributivgesetz: Verteilungsgesetze (Ausklammern/Ausmultiplizieren)


\subsection{Teilbarkeitsregeln (natürliche Zahlen)}
Ist a durch t ohne Rest teilbar?\\
$t|a$ t teilt a 


$t=2$ wenn die letzte Ziffer eine durch 2 teilbare Zahl darstellt\\
$t=3$ wenn die Quersumme durch 3 teilbar ist\\
$t=4$ wenn die letzten \textbf{zwei} Ziffern eine durch 4 teilbare Zahl bilden\\
$t=5$ wenn die letzte Ziffer eine durch 5 teilbare Zahl darstellt\\
$t=6$ wenn die Zahl durch 2 und 3 teilbar ist\\
$t=7$ (Für die Zahl 7 gibt es keine einfache Teilbarkeitsregel!)\\
$t=8$ wenn die letzten \textbf{drei} Ziffern eine durch 8 teilbare Zahl bilden\\
$t=9$ wenn die Quersumme durch 9 teilbar ist\\
$t=10$ wenn die letzte Ziffer eine 0 ist

Komplexere Regeln \\
$t=7$ wenn die alternierende 3er-Quersumme durch 7 teilbar ist\\
$t=11$ wenn die alternierende Quersumme durch 11 teilbar ist\\
$t=13$ wenn die alternierende 3er-Quersumme durch 11 teilbar ist


\subsection{Quersummenregeln}
\subsubsection{Quersumme}
$Q(1234)=4+3+2+1=10$

\subsubsection{Alternierende Quersumme}

$Q^{\prime}(1234)=4-3+2-1=2$\\
$Q^{\prime}(4321)=1-2+3-4=-1$\\
$Q^{\prime}(3223)=3-2+2-3=0$\\
$Q^{\prime}(68786979)=9+9+8+8-(7+6+7+6)=34-26=8$





\subsection{Ungleichungen}
\subsection{Gleichungen}
\subsubsection{Linieare Gleichungen}
\paragraph{Determinanten}
\paragraph{Cramer'sche Regel}
\subsection{Quadratische Gleichungen}
Allgemeine Form:
$ ax^2 + bx + c = 0 $

Normalform:
$ x^2 + px + q = 0 $ mit $p=\frac{b}{a}$ und $q=\frac{c}{a}$

Lösung: $x_{1,2}= - \frac{p}{2} \pm \sqrt[]{( \frac{p}{2})^2 -q }$

Linearfactoren (Produktform quadratischer Terme): $(x-x_1)\cdot (x-x_2 )=0$

\subsection{Polynomfunktion n-ten Grades}
Ein Polynom n-ten Grades lässt sich maximal in n Linearfaktoren zerlegen und hat maximal n Lösungen bzw. hat maximal n Nullstellen.
\begin{align*}
f(x) &= a_nx^n+a_{n-1}x^{n-1}+...+a_1x^1+a_0 \\
p(x) &=a_n(x-n_1)(x-n_2)(x-n_3)...(x-n_N) \textrm{ mit jeweils den Nullstellen } n_k
\end{align*}
