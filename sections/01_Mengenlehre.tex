% https://de.wikipedia.org/wiki/Liste_mathematischer_Symbole

\section{Mengenlehre}

\subsection{Mengenverknüpfung}

\begin{tabular}[h]{lll}

% The symbols \land, \lor and \lnot have synonyms: \wedge, \vee and \neg
Symbol &Menge & Formel \\
\hline
$\cup$ & Vereinigungsmenge	 & \{$x \in M \lor x \in N$\} \\
$\cap$ & Schnittmenge	 & \{$x \in M \land x \in N$\} \\
$\setminus$ & Differenzmenge	 & \{$x \in M \land x \notin N$\} \\
$\triangle$ & Symmetrische Differenz & \{TBD\} \\
$\times$ & Kartesisches Produkt & \{TBD\} \\
$\dot\cup$ & Vereinigung disjunkter Mengen  & \{TBD\} \\
$\sqcup$ & Disjunkte Vereinigung der Mengen  & \{TBD\} \\
$\bar{A}$ & Komplement & \{TBD\} \\
$\mathcal{P}$ & Potenzmenge & \{TBD\} \\
\end{tabular}

\subsection{Characteristische Mengen}

\begin{tabular}[h]{lll}
Symbol &Menge & Beispiele \\
\hline
$\emptyset$ & leere Menge & \{\} \\
$\mathbb{P}$ & Primzahlen & \{2,3,5,7,11,13,...\} \\
$\mathbb{N}$ & Natürliche Zahlen & \{1,2,3,...\} \\
$\mathbb{N}_0$ & Natürliche Zahlen mit 0 & \{0,1,2,3,..\} \\
$\mathbb{Z}$ & Ganze Zahlen & \{..,-3,-2,-1,0,1,2,3,...\} \\
$\mathbb{Z}^+$ & Ganze Zahlen & \{1,2,3,..\} \\
$\mathbb{Z}^-$ & Ganze Zahlen & \{...,-3,-2,-1\} \\
$\mathbb{Q}$ & Rationale Zahlen & \{ $\frac{p}{q} | (p \in \mathbb{Z}) \wedge (q \in  \mathbb{Z} \setminus \{0 \} ) $ \} \\
$\mathbb{I}$ & Irrationale Zahlen & \{$\sqrt{2},\pi, e,...$\} \\
$\mathbb{R}$ & Reelle Zahlen & $\mathbb{R} \cup \mathbb{I}$\\
$\mathbb{T}$ & Transzendente Zahlen & \{ $\pi, e$ \} \\
$\mathbb{C}$ & Komplexe Zahlen & \{ $x + iy | x,y \in \mathbb{R} $\} \\
\end{tabular}

\subsection{Teilbarkeitsregeln (natürliche Zahlen)}
Ist a durch t ohne Rest teilbar?\\
$t|a$ t teilt a


$t=2$ wenn die letzte Ziffer eine durch 2 teilbare Zahl darstellt\\
$t=3$ wenn die Quersumme durch 3 teilbar ist\\
$t=4$ wenn die letzten \textbf{zwei} Ziffern eine durch 4 teilbare Zahl bilden\\
$t=5$ wenn die letzte Ziffer eine durch 5 teilbare Zahl darstellt\\
$t=6$ wenn die Zahl durch 2 und 3 teilbar ist\\
$t=7$ (Für die Zahl 7 gibt es keine einfache Teilbarkeitsregel!)\\
$t=8$ wenn die letzten \textbf{drei} Ziffern eine durch 8 teilbare Zahl bilden\\
$t=9$ wenn die Quersumme durch 9 teilbar ist\\
$t=10$ wenn die letzte Ziffer eine 0 ist

Komplexere Regeln \\
$t=7$ wenn die alternierende 3er-Quersumme durch 7 teilbar ist\\
$t=11$ wenn die alternierende Quersumme durch 11 teilbar ist\\
$t=13$ wenn die alternierende 3er-Quersumme durch 11 teilbar ist


\subsection{Quersummenregeln}
\subsubsection{Quersumme}
$Q(1234)=4+3+2+1=10$

\subsubsection{Alternierende Quersumme}

$Q^{\prime}(1234)=4-3+2-1=2$\\
$Q^{\prime}(4321)=1-2+3-4=-1$\\
$Q^{\prime}(3223)=3-2+2-3=0$\\
$Q^{\prime}(68786979)=9+9+8+8-(7+6+7+6)=34-26=8$




\subsection{Intervalle}

\begin{align*}
[a,b]  &=  \{x| a \leq x \leq b, x \in \mathbb{R}\} \\
(a,b) &= ]a,b[ =  \{x| a <  x < b, x \in \mathbb{R}\} \\
[a,b) &= [a,b[ =  \{x| a \leq  x < b, x \in \mathbb{R}\} \\
(a,b] &= ]a,b] =  \{x| a <  x \leq b, x \in \mathbb{R}\}
\end{align*}